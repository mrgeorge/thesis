\chapter{Spectroscopic Weak Lensing}

\label{chap:speclens}

%-------- ABSTRACT  ---------------------
  
 
%--------------------------------------------------------------
% INTRODUCTION
%--------------------------------------------------------------

\section{Introduction}

We explore a technique first described by \citet{Blain2002} and
further elucidated by \citet{Morales2006} to use distortions in
measured velocity maps of galaxies as a probe of weak lensing by
foreground structure. Traditional weak lensing measurements seek to
constrain the shear induced on galaxy images and are limited by the
fact that galaxies have a broad range of intrinsic shapes and only the
lensed shapes are observed. Measurement of the velocity field of stars
or gas in the source galaxy can provide information about the
intrinsic shape and orientation of the galaxy, with a potential for
dramatically reducing the intrinsic shape noise. Such a measurement
translates into a great improvement of signal to noise for a given
lensing survey, or alternatively, allows for a modified survey
strategy with fewer source galaxies to achieve a given signal to
noise. An additional benefit to this approach is that potential
systematic errors are very different from traditional shear
measurements, providing cross-checks for possible biases due to errors
in photometric redshifts or shape measurements for example.

This work extends that of \citet{Blain2002} and \citet{Morales2006} by
incorporating the Tully-Fisher \citep{Tully1977} relation as a prior
on the amplitude of the velocity field which enables tighter
constraints on the intrinsic shapes of galaxies with limited data. We
also present a survey strategy using dithered pointings of a
multi-object optical spectrograph to coarsely map the velocity fields
of many galaxies simultaneously.

\section{The Idea}

Outline the principle of measuring velocity fields, incorporating TF
relation, inferring shapes and shears with velocity field only /
velocity field plus imaging.

% **** FIG *****
\begin{figure*}[htb]
%\epsscale{1.2}
\plottwo{speclens/fig1a}{speclens/fig1b}
\caption{Velocity maps for a galaxy with $PA=20\degr, b/a=0.3$. Left is unsheared, right is sheared $(e1,e2)=(0,0.3)$.}
\label{speclens_fig:vmap}
\end{figure*}
% **** FIG *****
 

\section{Simulations}

Galaxy image, PSF, shear, velocity maps, fiber integration.

Fitting models to observables.

% **** FIG *****
\begin{figure*}[htb]
%\epsscale{1.2}
\plottwo{speclens/fig2a}{speclens/fig2b}
\caption{Image (left) and flux-weighted, PSF-convolved velocity map
  (right) for a galaxy with $PA=20\degr, b/a=0.3$ and Gaussian seeing
  with $\rm{FWHM}=1.5\arcsec$. The overlaid circles show a 7-point dither
  pattern of fibers with radius $1\arcsec$.}
\label{speclens_fig:fiber_pattern}
\end{figure*}
% **** FIG *****

% **** FIG *****
\begin{figure*}[htb]
%\epsscale{1.2}
\plotone{speclens/fig3}
\caption{Line of sight velocity in a ring of radius $2\arcsec$
  corresponding to the centers of the outer fiber positions shown in
  Figure~\ref{speclens_fig:fiber_pattern}. Model parameters are those shown in
  the legend, with a fiducial model followed by other models with one
  parameter altered. Error bars show expected velocity uncertainties
  ($30~{\rm km/s}$) measured at the positions of the six outer fibers.}
\label{speclens_fig:observable}
\end{figure*}
% **** FIG *****


\section{Survey Plan}

Flux limit, sample density, estimated S/N.

Cluster field, galaxy-galaxy lensing, cosmic shear??

Based on GalSim package\footnote{Available at \url{https://github.com/GalSim-developers/GalSim}}

\section{Caveats}

Intrinsic disk ellipticity (see \citealt{Franx1992, Franx1994}), OII
distribution, uncertain velocity profiles, uncertain TF evolution, PSF
variation across pointings.

















\section{Shear and Galaxy Kinematics}
\label{speclens_s:theory}

\section{Observational Effects}

\section{Prospects}
