\chapter{Introduction}
\label{chap:intro}

\begin{quote}{\textit{``A thesis is not supposed to be your best work; it's just
  practice.''}

--- Marc Davis}
\end{quote}


A multitude of observations of our Universe, from the anisotropies of
the cosmic microwave background to the clustering of galaxies and the
expansion history derived from supernovae, can be described by a
simple yet befuddling cosmological model. In this model the familiar
baryonic component of the Universe, which can be detected through
electromagnetic interactions, is dwarfed by two dark constituents that
dominate the energy density. Dark matter is effectively collisionless
and nonrelativistic, seeding structure growth and hosting galaxies in
deep gravitational potentials called halos. Perturbations in the
density of baryons and dark matter, initialized after a period of
rapid inflation, eventually collapsed under gravitational instability
to form these galaxies and halos which merge and grow
hierarchically. Dark energy exerts a uniform negative pressure,
accelerating expansion and suppressing structure growth at late
times. The existence of these dark components implies that extensions
are necessary to our standard models of particle physics, gravity, or
both.

Testing the validity of this cosmological model and constraining its
parameters is the focus of much observational and theoretical effort.
What is dark matter and how is it distributed?  Does it
couple with baryons or self-interact in detectable ways?  Does the
dark energy density vary with space or time? Does general
relativity accurately describe gravity on large and small scales?
These are the questions we ultimately aim to address to further our
understanding of the dark side of our Universe.


Dark matter

-what is it, how is it distributed

-how does it impact galaxy properties


Galaxy evolution

-how did we come to have a universe with menagerie of galaxies

-why are there different types, bimodal distributions

-what is their fate

-physical processes responsible for differences

-address argument that denser halos have higher growth rate


Analysis

-clustering, lensing, multiwavelength


Surveys

-from Lick to CfA to SDSS, DEEP2, COSMOS and onward


Framework

-LCDM

-halo model


understanding galaxies can play a supporting role in cosmological
studies, but raises interesting questions in its own right. Why do
galaxies look the way they do, what caused them to look that way?

This thesis presents a systematic study of the relationships between
galaxy properties and their environment, to both address the basic
questions about galaxy evolution and to support further cosmological
study with a better understanding of how galaxies inhabit dark matter
halos.

Outline of chapters:
Chapter~\ref{chap:catalog} - group catalog.
Chapter~\ref{chap:centering} - lensing study of halo centers.
Chapter~\ref{chap:transformers} - galaxy properties vs groupcentric
distance.
Chapter~\ref{chap:outlook} - outlook beyond continuing similar
analyses with new surveys; new techniques for mapping dark matter
(spectroscopic lensing and peculiar velocities).
