\chapter{Introduction}
\label{chap:intro}

%\begin{quote}{\textit{``A thesis is not supposed to be your best work; it's just
%  practice.''}
%
%--- Marc Davis}
%\end{quote}

A multitude of observations of our Universe, from the anisotropies of
the cosmic microwave background to the clustering of galaxies and the
expansion history derived from supernovae, can be described by a
simple yet befuddling cosmological model. In this model the familiar
baryonic component of the Universe, which can be detected through
electromagnetic interactions, is dwarfed by two dark constituents that
dominate the energy density. Dark matter is effectively collisionless
and nonrelativistic, seeding structure growth and hosting galaxies in
deep gravitational potentials called halos. Perturbations in the
density of baryons and dark matter, initialized after a period of
rapid inflation, eventually collapsed under gravitational instability
to form these galaxies and halos which merge and grow
hierarchically. Dark energy exerts a uniform negative pressure,
accelerating expansion and suppressing structure growth at late
times. The existence of these dark components implies that extensions
are necessary to our standard models of particle physics, gravity, or
both.

Testing the validity of this cosmological model and constraining its
parameters is the focus of much observational and theoretical effort.
What is dark matter and how is it distributed in space?  Does it
couple with baryons or self-interact in detectable ways?  Does the
dark energy density vary with space or time? Does general
relativity accurately describe gravity on large and small scales?
These are the questions we ultimately aim to address to further our
understanding of the dark side of our Universe.

On large scales, the distribution of matter is accurately described by
the linear evolution of density perturbations from inflation under
gravitational instability. On smaller scales, density fluctuations
become nonlinear and collapse into halos. Halos are densest near their
centers and may host a central galaxy, and more massive halos can host
additional satellite galaxies that sample the dark matter
density as a stochastic tracer. Numerical simulations reveal a wide
variety of halo shapes and density profiles for these halos, but on
average they are spherical and can be described with a simple model
the depends on their mass and concentration. A halo model with these
assumptions can remarkably reproduce many observations of
nonlinear structure.

There is still much to be learned from constraining the parameters of
this cosmological model and determining where the simplifying
description of halos breaks down. The growth of structure can be
measured from the distribution of halo masses and its evolution over
time. This offers a test of dark energy that is complementary to
measurements of the expansion history.  The detailed density profiles
in halos and abundance of substructure are sensitive to the mass and
interaction cross section of dark matter, which ultimately tell us
about the physics of its formation.

Complications
-measuring halo masses
-identifying central/satellite galaxies
-bias = how galaxies populate halos, dependence on galaxy type

Observables:
-clustering, lensing, multiwavelength
-Weak lensing is a powerful tool to study galaxies and dark matter
halos since it is sensitive to the total mass distribution and can
probe large and diverse samples of galaxies. 
-importance of small scales for max info content of large surveys


To study dark matter we must understand galaxies, which pose questions
in their own right.
understanding galaxies can play a supporting role in cosmological
studies, but raises interesting questions in its own right. Why do
galaxies look the way they do, what caused them to look that way?

Galaxy evolution

-how did we come to have a universe with menagerie of galaxies

-why are there different types, bimodal distributions

-what is their fate

-physical processes responsible for differences

-address argument that denser halos have higher growth rate

Understanding the mass distribution around
galaxies requires disentangling the contributions from stars, gas, and
dark matter.


Surveys

-from Lick to CfA to SDSS, DEEP2, COSMOS and onward


understanding galaxies can play a supporting role in cosmological
studies, but raises interesting questions in its own right. Why do
galaxies look the way they do, what caused them to look that way?




This thesis presents a systematic study of the relationships between
galaxy properties and their environment, to both address the basic
questions about galaxy evolution and to support further cosmological
study with a better understanding of how galaxies inhabit dark matter
halos.



I have connected measurements of X-ray
emission and weak lensing for a large sample of galaxy groups spanning
nearly half the age of the Universe \citep{George2011}. This sample
consists of the most massive halos in the largest contiguous area
mapped by the Hubble Space Telescope, and our team added new
optical spectroscopy of group members which I analyzed. My colleagues
and I have exploited the high-resolution imaging and multi-wavelength photometry from the
COSMOS survey to constrain dark
matter halo properties of this group sample with weak lensing \citep{Leauthaud2010,
  George2012, Schmidt2012, Ford2012, Taylor2012} and also for studying
the masses, colors, and morphologies of galaxies residing in these
halos \citep{George2011, George2013, Leauthaud2012b}.
Miscentering is a dominant systematic uncertainty
in measuring halo masses \citep{Rozo2011}, so identifying
correct centers and modeling offsets is crucial. To address this
issue, I developed a new approach to find halo centers based on the
stacked weak lensing signal \citep{George2012}. The models I
developed can readily be extended to measure substructure around
satellites and the stellar mass to light ratio in galaxies with
lensing and dynamical measurements.

The properties of galaxies are linked to their environment, and
quantifying these correlations is important to understand the
coevolution of baryonic and dark matter. While it has long been known
that dense environments host galaxies with greater masses, lower
star-formation rates, and more bulge-dominated morphologies
\citep[see][for a review]{Blanton2009}, the processes that produce these
correlations remain unclear, in part due to the many interrelated
variables at play. Large data sets from recent and upcoming surveys
are finally enabling us to isolate trends in specific properties with
others fixed. My recent study of group member galaxies quantifies
the separate dependences of galaxy color and morphology on
group-centric distance (at effectively fixed stellar mass, halo mass,
and redshift), which constrains models of galaxy evolution involving
gas stripping, feedback, and mergers \citep{George2013}.


Outline of chapters:
Chapter~\ref{chap:catalog} - group catalog.
Chapter~\ref{chap:centering} - lensing study of halo centers.
Chapter~\ref{chap:transformers} - galaxy properties vs groupcentric
distance.
Chapter~\ref{chap:outlook} - outlook beyond continuing similar
analyses with new surveys; new techniques for mapping dark matter
(spectroscopic lensing and peculiar velocities).







Dark matter is a key component of the standard cosmological model, a
necessary scaffolding for the growth of galaxies and large-scale
structure. But its properties are not understood. 
