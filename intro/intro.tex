\chapter{Introduction}
\label{chap:intro}

%\begin{quote}{\textit{``A thesis is not supposed to be your best work; it's just
%  practice.''}
%
%--- Marc Davis}
%\end{quote}

A multitude of observations of our Universe, from the anisotropies of
the cosmic microwave background to the clustering of galaxies and the
expansion history derived from supernovae, can be described by a
simple yet befuddling cosmological model. In this model the familiar
baryonic component of the Universe, which can be detected through
electromagnetic interactions, is dwarfed by two dark constituents that
dominate the energy density. Dark matter is effectively collisionless
and nonrelativistic, seeding structure growth and hosting galaxies in
deep gravitational potentials called halos. Perturbations in the
density of baryons and dark matter, initialized after a period of
rapid inflation, eventually collapsed under gravitational instability
to form these galaxies and halos which merge and grow
hierarchically. Dark energy exerts a uniform negative pressure,
accelerating expansion and suppressing structure growth at late
times. The existence of these dark components implies that extensions
are necessary to our standard models of particle physics, gravity, or
both.

Testing the validity of this cosmological model and constraining its
parameters is the focus of much observational and theoretical effort.
What is dark matter and how is it distributed in space?  Does it
couple with baryons or self-interact in detectable ways?  Does the
dark energy density vary with space or time? Does general
relativity accurately describe gravity on large and small scales?
These are the questions we ultimately aim to address to further our
understanding of the dark side of our Universe.

On large scales, the distribution of matter is accurately described by
the linear evolution of density perturbations from inflation under
gravitational instability. On smaller scales, density fluctuations
become nonlinear and collapse into halos. Halos are densest near their
centers and may host a central galaxy, and more massive halos can host
additional satellite galaxies that sample the dark matter
density as a stochastic tracer. Numerical simulations reveal a wide
variety of halo shapes and density profiles for these halos, but on
average they are spherical and can be described with a simple model
the depends on their mass and concentration. A halo model with these
assumptions can remarkably reproduce many observations of
nonlinear structure.

There is still much to be learned from constraining the parameters of
this cosmological model and determining where the simplifying
description of halos breaks down. The growth of structure can be
measured from the distribution of halo masses and its evolution over
time. This offers a test of dark energy that is complementary to
measurements of the expansion history.  The detailed density profiles
in halos and abundance of substructure are sensitive to the mass and
interaction cross section of dark matter, which ultimately tell us
about the physics of its formation.

Complications arise when constraining these quantities from
observations. Galaxy surveys have served as the workhorse for
large-scale structure measurements. Wide area imaging taken at Lick
Observatory in the 1950s provided the first systematic probe of the
large-scale spatial distribution of galaxies. Later the CfA Redshift
Survey added a third dimension to such maps, and opened the field for
a proliferation of modern experiments including the Sloan Digital Sky
Survey and DEEP2, pushing boundaries in volume and depth of
coverage. Multi-wavelength imaging and spectroscopic surveys such as
COSMOS have supplemented these optical studies with observations
from radio to X-ray bands, providing new insight into the interplay of
stars and gas with dark matter.

Dark matter halo masses and density profiles can only be measured from
these data indirectly. Galaxies are visible tracers of the dark matter
density field, but the details of how galaxies populate dark matter
halos can be complicated. From two- or three-dimensional maps, groups
and clusters of galaxies can be identified based on their
proximity. Many estimators have been employed to extract halo masses
from observables. Galaxy counts, luminosities, and stellar masses can
be derived relatively inexpensively, while it is generally more costly
to measure galaxy dynamics or the temperature and luminosity profiles
of the hot gas in massive halos. Gravitational lensing offers another
probe of the projected matter distribution that is in principle
independent of the baryonic tracers used in other estimators. In
general, these methods have been shown to correlate with one another
and produce a coherent picture of halo masses and density profiles.
But for precision cosmology, it is essential to probe these small
nonlinear scales and to improve our understanding of the biases in
techniques for extracting the matter distribution from observations.

The mission to understand how galaxies trace dark matter plays in
important role in cosmological studies, but also raises interesting
questions in its own right. Galaxy properties are broadly bimodal,
with a population of more massive and luminous red galaxies that tend
to have older stellar populations, more spheroidal morphologies, and
concentrated light profiles than a distinct population of younger,
blue star-forming galaxies. Why do galaxies look the way they do, and
what processes led to the menagerie we observe?

This thesis presents a systematic study of the relationships between
galaxy properties and their environment, to both address the basic
questions about galaxy evolution and to support further cosmological
study with a better understanding of how galaxies inhabit dark matter
halos. I have connected measurements of X-ray emission and weak
lensing for a large sample of galaxy groups spanning nearly half the
age of the Universe \citep{George2011}. This sample consists of the
most massive halos in the largest contiguous area mapped by the Hubble
Space Telescope, and and includes new optical spectroscopy of group
members. This group sample has been exploited with the high-resolution
imaging and multi-wavelength photometry from the COSMOS survey to
constrain dark matter halo properties of this group sample with weak
lensing \citep{Leauthaud2010, George2012, Schmidt2012, Ford2012,
  Taylor2012} and also for studying the masses, colors, and
morphologies of galaxies residing in these halos \citep{George2011,
  George2013, Leauthaud2012b}.  Miscentering is a dominant systematic
uncertainty in measuring halo masses \citep{Rozo2011}, so identifying
correct centers and modeling offsets is crucial. To address this
issue, I developed a new approach to find halo centers based on the
stacked weak lensing signal \citep{George2012}. The models I developed
can readily be extended to measure substructure around satellites and
the stellar mass to light ratio in galaxies with lensing and dynamical
measurements.

The properties of galaxies are linked to their environment, and
quantifying these correlations is important to understand the
coevolution of baryonic and dark matter. While it has long been known
that dense environments host galaxies with greater masses, lower
star-formation rates, and more bulge-dominated morphologies
\citep[see][for a review]{Blanton2009}, the processes that produce these
correlations remain unclear, in part due to the many interrelated
variables at play. Large data sets from recent and upcoming surveys
are finally enabling us to isolate trends in specific properties with
others fixed. This study of group member galaxies quantifies
the separate dependences of galaxy color and morphology on
group-centric distance (at effectively fixed stellar mass, halo mass,
and redshift), which constrains models of galaxy evolution involving
gas stripping, feedback, and mergers \citep{George2013}.

The work in this thesis has been published in three separate papers,
each of which is translated into a chapter
here. Chapter~\ref{chap:catalog} \citep{George2011} describes the
construction of the group catalog and detailed tests of group
membership assignment using simulations. Chapter~\ref{chap:centering}
\citep{George2012} presents a weak lensing study to test different
tracers of halo centers and their impact on halo mass
estimates. Chapter~\ref{chap:transformers} \citep{George2013} employs
those halo centers to measure the dependence of galaxy properties on
groupcentric distance, illustrating how star-formation rates and
structural parameters in galaxies are related to the dark matter halos
they inhabit.
%Chapter~\ref{chap:outlook} - outlook beyond continuing similar
%analyses with new surveys; new techniques for mapping dark matter
%(spectroscopic lensing and peculiar velocities).
